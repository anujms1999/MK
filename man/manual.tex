\documentclass[12pt]{amsart}
\usepackage{amsmath}
\usepackage{setspace}
\usepackage{graphicx}
\usepackage{hyperref}
\usepackage{cite}
\usepackage{subfigure}
\usepackage{listings}
\usepackage{color}
%\usepackage[nomarkers,figuresonly]{endfloat}
\usepackage{geometry} % see geometry.pdf on how to lay out the page. There's lots.
\usepackage{paralist}
\usepackage{lscape}
\usepackage{caption}
\usepackage{lineno}
%\linenumbers

\usepackage{pgf}
\usepackage{tikz}
\usepackage{pgfplots}
%\usepackage{harvard}
\usetikzlibrary{shapes,arrows,chains,automata,fit}
\usetikzlibrary{positioning}
\usetikzlibrary{shapes.geometric,intersections}

\author{Youngung Jeong}
\address{International Center for Automotive Research\\
  Clemson University
}
\email[Y. Jeong]{youngung.jeong@nist.gov}

%\usepackage{authblk}
\geometry{a4paper} % or letter or a5paper or ... etc
% \geometry{landscape} % rotated page geometry
\hypersetup{
  colorlinks,%
  citecolor=blue,%,black
  filecolor=blue,%
  linkcolor=blue,%
  urlcolor=blue
}

\doublespacing

%% Custom colors
\definecolor{dkgreen}{rgb}{0,0.6,0}
%\definecolor{lime}(0,255,0)
\definecolor{gray}{rgb}{0.5,0.5,0.5}
\definecolor{mauve}{rgb}{0.58,0,0.82}
\definecolor{lightgray}{rgb}{0.83, 0.83, 0.83}
\definecolor{lightergray}{rgb}{0.90, 0.90, 0.90}
\definecolor{verylightgray}{rgb}{0.95, 0.95, 0.95}

%% tikz customization
\tikzset{state/.style={rectangle,rounded corners,draw=black, very thick,
    minimum height=2em,inner sep=2pt,text centered}}
\tikzset{decision/.style={diamond,aspect=2,draw=black,very thick,
    minimum height=2em,inner sep=2pt,text centered}}
\tikzset{process/.style={circle,draw=black,very thick,
    minimum height=2em,inner sep=2pt,text centered}}
\tikzset{dot/.style={circle,draw=black,thick,
    inner sep=0pt,minimum size=4pt}}

\title{Manual for MK}

\begin{document}
\pagenumbering{arabic}
\newpage
\maketitle
\newpage
\setcounter{tocdepth}{1}
\tableofcontents

\section{Introduction}
\label{sec:intro}

\section{Theoretical backgrounds}
\label{sec:theory}

% The material is considered rigid-plastic.
% The fundamental constitutive equation considered in this study is a function of strain rate ($\varepsilon_{ij}$).
% The accumulative strain results from an integration of the history:
% \begin{eqnarray}
%   \label{eq:accEPS}
%   \varepsilon_{ij} = \int_{t} \dot{\varepsilon}_{ij} dt
% \end{eqnarray}
% Assuming isotropic hardening through

\section{hardening}
Flow stress ($\Sigma^f$) and its rate sensitivity is described such that
\begin{equation}
\Sigma^f=\sigma(\varepsilon)(\frac{\dot{\varepsilon}}{\dot{\varepsilon_0}})^m
\end{equation}

\section{Yield functions}
\label{sec:yieldfunctions}
\subsection{Von Mises yield criterion}
\label{sec:vm}
Von Mises yield criterion:

\begin{eqnarray}
  \label{eq:vonMises}
  \phi=h^{\frac{1}{2}}
\end{eqnarray}
where h is
\begin{eqnarray}
  \label{eq:vonMises2}
  h &=& \frac{1}{2} [(\sigma_{11}-\sigma_{22})^2+\sigma_{11}^2+\sigma_{22}^2+6\sigma_{12}^2]
\end{eqnarray}
The first derivative of the von Mises yield criterion is expressed as:

\begin{eqnarray}
  \label{eq:vonMisesDphi}
  \frac{\partial \phi}{\partial\sigma_{ij}}&=& \frac{1}{2}\frac{\partial h}{\partial \sigma_{ij}} h^{-\frac{1}{2}}\\
                                          &=&\frac{1}{2}\frac{\partial h}{\partial \sigma_{ij}} \frac{1}{h^\frac{1}{2}}\\
                                          &=&\frac{1}{2\phi}\frac{\partial h}{\partial \sigma_{ij}}\\
                                          &=&\frac{1}{2}\phi^{-1} \frac{\partial h}{\partial \sigma_{ij}}
                                        \end{eqnarray}
where only three components of $\frac{\partial h}{\partial \sigma_{ij}}$ are relevant in the plane-stress space, that is
\begin{eqnarray}
  \label{eq:dh}
  \frac{\partial h}{\partial \sigma_{11}}&=&2\sigma_{11}-\sigma_{22} \\
  \frac{\partial h}{\partial \sigma_{22}}&=&-\sigma_{11}+2\sigma_{22}\\
  \frac{\partial h}{\partial \sigma_{12}}&=&6\sigma_{12}
\end{eqnarray}
Therefore, according to Eq \ref{eq:vonMisesDphi}, only three components of $\frac{\partial \phi}{\partial\sigma_{ij}}$ are relevant:
\begin{eqnarray}
  \label{eq:vonMisesDphi}
  \frac{\partial \phi}{\partial\sigma_{11}}&=&\frac{1}{2\phi}\frac{\partial h}{\partial \sigma_{11}}=\frac{1}{2\phi} (2\sigma_{11}-\sigma_{22})\\
  \frac{\partial \phi}{\partial\sigma_{22}}&=&\frac{1}{2\phi}\frac{\partial h}{\partial \sigma_{22}}=\frac{1}{2\phi} (2\sigma_{22}-\sigma_{11})\\
  \frac{\partial \phi}{\partial\sigma_{12}}&=&\frac{1}{2\phi}\frac{\partial h}{\partial \sigma_{12}}=\frac{1}{2\phi} (6\sigma_{12})
\end{eqnarray}
The second order derivative is obtained as follows.
\begin{eqnarray}
  \label{eq:d2phiVM}
  \frac{\partial^2\phi}{\partial\sigma^2}&=&\frac{  \partial{(\frac{\partial \phi}{\partial\sigma_{ij}})  }     }{\partial \sigma_{kl}}\\
  &=&\frac{ \frac{1}{2}  \partial{(  h^{-\frac{1}{2}}\frac{\partial h}{\partial \sigma_{ij}}   )}}{\partial \sigma_{kl}}\\
  &=& 0.5  \frac{\partial(h^{-\frac{1}{2}})}{\partial\sigma_{kl}} \frac{\partial h}{\partial \sigma_{ij}}   + 0.5 h^{-\frac{1}{2}}  \frac{\partial(  \frac{\partial h}{\partial \sigma_{ij}}   )}{\partial\sigma_{kl}}\\
  &=& 0.5  (-1/2)h^{-3/2} \frac{\partial h}{\partial\sigma_{kl}}  + 0.5 h^{-\frac{1}{2}}  \frac{\partial(  \frac{\partial h}{\partial \sigma_{ij}}   )}{\partial\sigma_{kl}}\\
  &=& (-1/4) (h^{-1/2})^3 \frac{\partial h}{\partial\sigma_{kl}}  + 0.5 h^{-\frac{1}{2}}  \frac{\partial(  \frac{\partial h}{\partial \sigma_{ij}}   )}{\partial\sigma_{kl}}\\
  \frac{\partial^2\phi}{\partial\sigma^2}_{ij,kl}&=& - \frac{1}{4\phi^3} \frac{\partial h}{\partial\sigma_{kl}}  + \frac{1}{2\phi}  \frac{\partial(  \frac{\partial h}{\partial \sigma_{ij}}   )}{\partial\sigma_{kl}}
\end{eqnarray}
The above is implemented in \verb|for.f| as \verb|subroutine vm|.


\subsection{Hill yield criterion}
\subsubsection{Quadratic plane-stress condition}
Hill's quadratic yield criterion under the plane-stress condition is a function of only two r-values (i.e., $R_0$ and $R_{90}$).
It is written as
\begin{eqnarray}
  \label{eq:hillquad}
  \sigma_{1}^2 + \frac{R_0 (1 + R_{90})}{R_{90}(1+R_0)}\sigma_s^2-\frac{2R_0}{1+R_0}\sigma_1\sigma_2=(\sigma_1^y)^2.
\end{eqnarray}
where $\sigma^y_1$ refers to the yield stress along axis 1 (i.e., the rolling direction of the material).

\subsubsection{Quadratic generalized condition including shear stresses}
Hill's generalized quadratic yield criterion is such that
\begin{equation}
  \label{eq:hillgenquad}
  \begin{aligned}
    \phi =& [F(\sigma_{22}-\sigma_{33})^2+G(\sigma_{33}-\sigma_{11})^2+\\
          &H(\sigma_{11}-\sigma_{22})^2+2L\sigma_{23}^2+2M\sigma_{31}^2+2N\sigma_{12}^2]^\frac{1}{2}.
  \end{aligned}
\end{equation}
In the plane-stress space, the above reduces to
\begin{equation}
  \label{eq:hillgenquad}
  \begin{aligned}
    \phi = [F\sigma_{22}^2+G\sigma_{11}^2+H(\sigma_{11}-\sigma_{22})^2+2N\sigma_{12}^2]^\frac{1}{2}.
  \end{aligned}
\end{equation}
Likewise in the derivation of von Mises yield function in section \ref{sec:vm}, h is defined as below:
\begin{eqnarray}
  \label{eq:hill48H}
  h&=&F\sigma_{22}^2+G\sigma_{11}^2+H(\sigma_{11}-\sigma_{22})^2+2N\sigma_{12}^2\\
   &=&(G+H)\sigma_{11}^2+(F+H)\sigma_{22}^2-2H\sigma_{11}\sigma_{22}+2N\sigma_{12}^2.
\end{eqnarray}
Like in Eq. \ref{eq:vonMisesDphi},  $\frac{\partial \phi}{\partial\sigma_{ij}}$ is a function of $\phi$ and $\frac{\partial h}{\partial \sigma_{ij}}$.
$\frac{\partial h}{\partial \sigma_{ij}}$ for Hill's yield function is expressed as:
\begin{eqnarray}
  \label{eq:h_Hill48}
  \frac{\partial h}{\partial \sigma_{11}}&=&2(G+H)\sigma_{11} -2H\sigma_{22}  \\
  \frac{\partial h}{\partial \sigma_{22}}&=&2(F+H)\sigma_{22} -2H\sigma_{11}   \\
  \frac{\partial h}{\partial \sigma_{12}}&=&4N\sigma_{12}
\end{eqnarray}
As a reminder,
\begin{eqnarray}
  \label{eq:vmdphi_gen}
  \frac{\partial \phi}{\partial\sigma_{ij}}=\frac{1}{2}\phi^{-1} \frac{\partial h}{\partial \sigma_{ij}}.
\end{eqnarray}
Thus, following is obtained:
\begin{eqnarray}
  \label{eq:vmdphi_components}
  \frac{\partial \phi}{\partial\sigma_{11}}&=&\frac{1}{2}\phi^{-1} (2(G+H)\sigma_{11} -2H\sigma_{22} )\\
  \frac{\partial \phi}{\partial\sigma_{22}}&=&\frac{1}{2}\phi^{-1} (2(F+H)\sigma_{22} -2H\sigma_{11} )\\
  \frac{\partial    h}{\partial\sigma_{12}}&=&\frac{1}{2}\phi^{-1} (4N\sigma_{12}).
\end{eqnarray}
Since the same relation between $h$ and $\phi$ holds as in Von Mises yield criterio, Eq. \ref{eq:d2phiVM} can be used to obtain the $2^{nd}$ derivatives for Hill 48 yield criterion as well.
\begin{eqnarray}
  \label{eq:d2phiHill}
  \frac{\partial^2\phi}{\partial\sigma^2}_{ij,kl}&=& - \frac{1}{4\phi^3} \frac{\partial h}{\partial\sigma_{kl}}  + \frac{1}{2\phi}  \frac{\partial(  \frac{\partial h}{\partial \sigma_{ij}}   )}{\partial\sigma_{kl}}
\end{eqnarray}

\subsection{Functions in FLD module}
In \verb|func_fld.py| module, \verb|func_fld1| provides an objective function and jacobian to determine the initial state of region B in response to the given state of region A.
The objective function is multivariate such that
\begin{eqnarray}
  \label{eq:objf_func_fld1}
  F_1&=& (\frac{\partial \phi^B}{\partial \boldsymbol{\sigma}^B_{22}})|_\text{band}\\
  F_2&=& \sigma_{12}^B|_\text{band}- \sigma_{11}^B|_\text{band} \frac{\sigma_{12}^A}{\sigma_{11}^A}|_\text{band}\\
  F_3&=& \phi^B- 1. \\
  F_4&=& \ln(f_0 \frac{\bar{\sigma}^B}{\bar{\sigma}^A}) + (m^A-m^B) \ln(qq^A) - d\lambda^B \frac{\partial \phi^B}{\partial \sigma^B_{11}}|_\text{band}
\end{eqnarray}
\begin{eqnarray}
  \label{eq:jacob_func_fld1}
  J_{11}=\frac{\partial(F_1)}{\partial(x_1)}&=&  \\
  J_{12}=\frac{\partial(F_1)}{\partial(x_2)}&=&   \\
  J_{13}=\frac{\partial(F_1)}{\partial(x_2)}&=&   \\
  J_{14}=0.                                 &&    \\
  J_{3}=\frac{\partial(F_1)}{\partial(x_1)}&=&    \\
  J_{4}=\frac{\partial(F_1)}{\partial(x_1)}&=&
\end{eqnarray}
In \verb|func_fld.py| module, another import function is \verb|func_fld2|.
\begin{eqnarray}
  \label{eq:dpsi}
  d\psi=\frac{ (\frac{\partial \phi}{\partial\sigma_{11}}-\frac{\partial \phi}{\partial\sigma_{22}}) \tan(b_9)}    {1+\tan^2(b_9)}\  \dot{t}
\end{eqnarray}


\verb|E| is defined as:
\begin{eqnarray}
  \label{eq:yancien}
  E = -y_3 - y_4 - deltat (\frac{\partial \phi^B}{\partial\sigma^B_{11}}+\frac{\partial \phi^B}{\partial\sigma^B_{22}})+y_1+y_2+x_0(\frac{\partial \phi^A}{\partial\sigma^A_{11}}+\frac{\partial \phi^A}{\partial\sigma^A_{22}})
\end{eqnarray}
\begin{equation}
  \label{eq:dxp}
  dx^p_1=2\cos\psi^{new}\sin\psi^{new}(\sigma_{22}^B|_{band}-\sigma_{11}^B|_{band}) \\
  - 2 (\cos^2\psi^{new}-\sin^2\psi^{new}) \sigma_{12}^B|_{band}
\end{equation}

\begin{equation}
  \label{eq:dxp}
  dx^p_2=2\cos\psi^{new}\sin\psi^{new}(\sigma_{22}^B|_{band}-\sigma_{11}^B|_{band}) \\
  - 2 (\cos^2\psi^{new}-\sin^2\psi^{new}) \sigma_{12}^B|_{band}
\end{equation}

\begin{equation}
  \label{eq:dxp}
  dx^p_6=(cos^2\psi-sin^2\psi^B) (\sigma^B_{11}|_{band}-\sigma^B_{22}|_{band}) - 4 \sin\psi^B \cos\psi^B
\end{equation}
\begin{eqnarray}
  \label{eq:dpe}
  dpe = (\frac{\partial \phi^A}{\partial\sigma^A_{11}}-\frac{\partial \phi^A}{\partial\sigma^A_{22}}) \tan\psi^{new} \frac{1}{1+\tan^2\psi^{new}}
\end{eqnarray}
\begin{eqnarray}
  \label{eq:func2_Fi}
  F_1&=&f_0 \bar{\sigma}^B\sigma_{11}^B|_{band}\ \exp(E) - \bar{\sigma}^A \sigma_{11}^A|_{band} (Q^{m^B})qq^{m^A-m^B}\\
  F_2&=&\sigma_{11}^B|_{band}\ \sigma_{12}^A|_{band}\ -\sigma_{12}^B|_{band}\ \sigma_{11}^A|_{band}\\
  F_3&=&\phi^B-1.\\
  F_4&=&deltat (\frac{\partial \phi^B}{\partial\sigma^B_{11}}|_{band}) - x_0 (\frac{\partial \phi^A}{\partial\sigma^A_{11}}_{band})
\end{eqnarray}

\section{examples}
\end{document}